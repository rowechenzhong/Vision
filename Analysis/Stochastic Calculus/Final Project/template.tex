\documentclass[12pt]{amsart}
\usepackage[margin=1in]{geometry}
\usepackage[tt=false]{libertine}
\usepackage{mathabx}
\usepackage[T1]{fontenc}
\usepackage[normalem]{ulem}
\usepackage[usenames,dvipsnames]{xcolor}
\usepackage[colorlinks=true]{hyperref}
\usepackage{enumerate}
\title{title of your project}
\author{your name}
\begin{document}
\maketitle
\begin{abstract}
A short (around one paragraph) summary of your writeup goes here.
\end{abstract}
\section{Introduction}
Introduce the topic, motivations, background, etc. You can have in mind your 18.676
classmates as your intended audience. Introduce some particulars of your topic that
you will explore in more detail in subsequent sections. This section might run
around 1 page.
\section{Another section}
Go into some aspect of the topic in more detail. This might be another 1-2 pages.
\subsection{A subsection}
\subsection{Another subsection}
\section{Another section (optional)}
Go into some other aspect of the topic in more detail. This might be another 1-2
pages.
\section{Conclusions}
A short summary of what you discussed and learned. This might be about half a
page.\bigskip
\subsection*{Further guidelines.}
\begin{itemize}
\item
The entire writeup should be at least 5 pages in this template, not including
bibliography. You are welcome to write more but 5 pages is the minimum.
\item You must include a bibliography and properly cite all references used.
\item A well written exposition should address both intuition and mathematical
formalism.
\item If you have additional questions, please get in touch!
\end{itemize}
\section*{References}
Don't forget to properly cite your sources!!!
\end{document}
