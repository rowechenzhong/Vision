\documentclass{scrartcl}
\usepackage{darkwhite}

\title{Stratonovich's Theory}
\subtitle{18.676 Spring 2024}
\date{Spring 2024}

\begin{document}

\maketitle

% (1.33) Exercise. All the processes considered below are bounded. For a subdivision L1 = (ti) of [0, t] and A E [0, 1], we set t/' = ti + A (ti+! - ti). 
% 10 ) If X = M + A, Y = N + B are two cont. semimarts. and H is a continuous 
% adapted process, set 
% K~ = LHti {(Xtt - Xti) (Ytt - Yti ) - (X, Y):t}. 
% i 
% Prove that 
% lim sup E [(K~)2] = 0. 
% 1L1140 A 
% 20 ) If A = 1 or if d (X, Y) is absolutely continuous with respect to the Lebesgue 
% measure, then 
% lim'" Ht(Xtl. - Xt.) (Ytl. - Yt.) = A t Hsd(X, Y)s 
% 1L1140 L:-t " ' , ' Jo , 
% in the L 2-sense. In the second case, the convergence is uniform in A. 
% 30 ) If F is a C1-function on JR, prove that 
% P-lim L(F(Xti+J-F(Xti ))2 = t F'(Xs)2d(X,X)s'


% (2.18) Exercise. Let X and Y be two continuous semimartingales. For a subdivision .1 of [0, t], a function f E C I (JR) and a probability measure fL on [0, 1], we 
% define 
% 144 Chapter IV. Stochastic Integration 
% 1°) Prove that 
% lim S~ = t f(Xs)dYs + Ii t f'(Xs)d(X, Y)s 
% 1.11---+0 10 10 
% in probability, with Ii = fOI S df.L(s). For f.L = 01/2 and f(x) = x, this limit 
% is called the Stratonovich integral of X against Y. For f.L = 00, we get the Ito 
% stochastic integral of f(X) against Y. Observe that this is the only integral for 
% which the resulting process is a local martingale when Y is a local martingale. 
% [Hint: Use Exercise (1.33).] 
% For f.L = 01, the limit is called the backward integral. 
% 2°) If we set 
% S~ = L (Yt;+1 - Yt;} t f (Xt;+S(I;+I-I;)) df.L(s) 
% i ~ 
% and if d (X, Y) is absolutely continuous with respect to the Lebesgue measure, 
% then S~ has the same limit as S~ in probability whenever 1.11 -+ O. 
% 3°) If OJ = Li fi (XI, ... , Xd) dXi is a closed differential form of class C I on 
% an open set U of]R.d and X = (Xl, ... , Xd ) a vector semimartingale with values 
% in U (i.e. P[3t : Xt (j. U] = 0) then 
% 1 --~ 11 i 1" 1t afi i j W - ~ fi(Xs)dX, + - ~ -(Xs)d(X , X )s 
% X(O,t) i=1 0 2 i,j 0 aXj 
% where X(O, t) is the continuous path (Xs(w), 0.::: s .::: t). We recall that the integral 
% of a closed form OJ along a continuous but not necessarily differentiable path 
% y : [0, t] -+ ]R.d is defined as n(y(t» - n(y(O» where n is a function such that 
% dn = OJ in a string of balls which covers y. 
% 4°) If B is the planar BM, express as a stochastic integral the area swept by 
% the line segment joining 0 to the point Bs , as s varies from 0 to t. 


% (3.15) Exercise. If X and Yare two continuous semimartingales, denote by 
% fot Xs odYs 
% the Stratonovich integral defined in Exercise (2.18). Prove that if F E C3 (lRd , lR) 
% and X = (Xl, ... , Xd ) is a vector semimartingale, then 
% it aF . F(Xt ) = F(Xo) + ~ -(Xs) 0 dX!


% (2.19) Exercise. We retain the notation of Exercise (2.18) in Chap. IV. 
% * 
% 1°) Let F be a holomorphic function in an open subset U of <C and i.V the 
% differential form F(z)dz. If Z is a conf. loco mart. such that P[3t ::::: ° : Zt ¢:. 
% U] = 0, then 
% ( i.V = t F(Zs)dZs = t F(Zs) 0 dZs a.s., 
% 1z(o.l) 10 10 
% where 0 stands for the Stratonovich integral. 
% 2°) In the situation of Theorem (2.11), we have 
% et - eo = ( i.V 
% 1z(o.l) 
% with i.V = (x 2 + lri (xdy - ydx) = Im(dz/z). 
% 3°) In the same situation, let St be the area swept by the segment [0, Zu], ° :::: u :::: t (see Exercise (2.18) in Chap. IV). Prove that there is a linear BM 
% 8 independent of p such that St = 8 At where At = ~ f~ p; d S. Another proof is 
% given in Exercise (3.10) Chap. IX which removes the condition a i= 0.

8.1.4. Exercises.
Exercise 8.1.8. In this exercise we will describe Itˆo’s approach to extending the validity of (8.1.6).
(i) The first step is to give another description of Stratonovich integrals.
Namely, given X, Y ∈ S(P; R), show that the Stratonovich integral of Y
with respect to X over the interval [0, T] is almost surely equal to
(*) lim
N→∞
2
XN −1
m=0
Y

(m+1)T
2N

+ Y
 mT
2N

2

X

(m + 1)T
2N

− X

mT
2N
 .
Thus, even if Y is not a semimartingale, we will say that Y : [0, T] × Ω −→
R is Stratonvich integrable on [0, T] with respect to X if the limit in (*)
exists in P-measure, in which case we will use R T
0
Y (t) ◦ dX(t) to denote this
limit. It must be emphasized that the definition here is “T by T” and not
simultaneous for all T’s in an interval.
(ii) Given an X ∈ S(P; R) and a T > 0, set Xˇ T
(t) = X

(T − t)
+

, and
suppose that
Xˇ T
(t), FˇT
t
, P

is a semimartingale relative to some filtration
{FˇT
t
: t ≥ 0}. Given a Y : [0, T] × Ω −→ R with the properties that
Y (· , ω) ∈ C

[0, T]; R

for P-almost every ω and that ω Y (t, ω) is Ft ∩FˇT
t
for each t ∈ [0, T], show that Y is Stratonovich integrable on [0, T] with
respect to X. In fact, show that
Z T
0
Y (t) ◦ dX(t) = 1
2
Z T
0
Y (t) dX(t) −
1
2
Z T
0
Y (t) dXˇ T
(t),
where each of the integrals on the right is the taken in the sense of Itˆo .
(iii) Let
β(t), Ft, P

be an R
n-valued Brownian motion. Given T ∈
(0,∞), set βˇT
(t) = β

(T − t)
+

, FˇT
t = σ

{βˇT
(τ ) : τ ∈ [0, t]}

, and
show that, for each ξ ∈ R
n,

(ξ, βˇT
(t))Rn , FˇT
t
, P

is a semimartingale with
hβˇT
, βˇT
i(t) = t ∧ T and bounded variation part t −
R t
0
βˇT
τ
T −τ
dτ . In particular, show that, for each ξ ∈ R
n and g ∈ C(R
n; R), t g

β(t)

is
Stratonovich integrable on [0, T] with respect to t

ξ, β(t)

Rn . Further, if
{gn}∞
1 ⊆ C(R
n; R) and gn −→ g uniformly on compacts, show that
Z T
0
gn

β(t)

◦ d

ξ, β(t)

Rn −→ Z T
0
g

β(t)

◦ d

ξ, β(t)

Rn
in P-measure.
(iv) Continuing with the notation in (iii), show that
f

β(T)

− f(0)) = Xn
i=1
Z T
0
∂if

β
228 8 Stratonovich’s Theory
for every f ∈ C
1
(R
n; R). In keeping with the comment at the end of (i), it
is important to recognize that although this form of Itˆo ’s formula holds for
all continuously differentiable functions, it is, in many ways less useful than
forms which we obtained previously. In particular, when f is no better than
once differentiable, the right hand side is defined only up to a P-null set for
each T and not for all T’s simultaneously.
(v) Let
β(t), Ft, P

be an R-valued Brownian motion, show that, for each
T ∈ (0,∞), t sgn
β(t)

is Stratonovich integrable on [0, T] with respect
to β, and arrive at

β(T)

 =
Z T
0
sgn
β(t)

◦ dβ(t).
After comparing this with the result in (6.1.7), conclude that the local time
`(T, 0) of β at 0 satisfies
Z T
0
|β(t)|
t
dt − `(T, 0) −

β(T)

 =
Z T
0
sgn
β(T − t)

dMˇ T
(t),
where Mˇ T
is the martingale part of βˇT
. In particular, the expression on the
left hand side is a centered Gaussian random variable with variance T.
Exercise 8.1.9. Itˆo’s formula proves that f ◦ Z ∈ S(P; R) whenever Z =
(Z1, . . . , Zn) ∈ S(P; R)
n and f ∈ C
2
(R
n; R). On the other hand, as Tanaka’s
treatment (cf. §6.1) of local time makes clear, it is not always necessary
to know that f has two continuous derivatives. Indeed, both (6.1.3) and
(6.1.7) provide examples in which the composition of a martingale with a
continuous function leads to a continuous semimartingale even though the
derivative of the function is discontinuous. More generally, as a corollary
of the Doob-Meyer Decomposition Theorem (alluded to at the beginning of
§ 7.1) one can show that f ◦Z will be a continuous semimartingale whenever
Z ∈ Mloc(P; R
n) and f is a continuous, convex function. Here is a more
pedestrian approach to this result.
(i) Following Tanaka’s procedure, prove that f ◦ Z ∈ S(P; R) whenever
Z ∈ S(P; R)
n and f ∈ C
1
(R
n; R) is convex. That is, if Mi denotes the
martingale part of Zi
, show that
f

Z(t)

−
Xn
i=1
Z t
0
∂if

Z(τ )

dMi(τ )
is the bounded variation part of 
Hint: Begin by showing that when A(t) ≡
hZi
, Zj i(t)

1≤i,j≤n
, A(t)−A(s)
is P-almost surely non-negative definite for all 0 ≤ s ≤ t, and conclude that
if f ∈ C
2
(R
n; R) is convex then
t
Xn
i,j=1
Z t
0
∂i∂jf

Z(τ )

hZi
, Zj i(dτ )
is P-almost surely non-decreasing.
(ii) By taking advantage of more refined properties of convex functions,
see if you can prove that f ◦ Z ∈ S(P; R) when f is a continuous, convex
function.
Exercise 8.1.10. If one goes back to the original way in which we described
Stratonovich in terms of Riemann integration, it becomes clear that the only
reason why we needed Y to be a semimartingale is that we needed to know
that
X
m≤2N t

∆N
mY
∆N
mX

converges in P-probability to a continuous function of locally bounded variation uniformly for t in compacts.
(i) Let Z =

Z1, . . . , Zn

∈ S(P; R)
n, and set Y = f ◦ Z, where f ∈
C
1
(R
n; R). Show that, for any X ∈ S(P; R),
X
m≤2N t

∆N
mY
∆N
mX

−→ Xn
i=1
Z t
0
∂if

Z(τ )

hZi
, Xi(dτ )
in P-probability uniformly for t compacts.
(ii) Continuing with the notation in (i), show that
Z t
0
Y (τ ) ◦ dX(τ ) ≡ lim
N→∞ Z t
0
Y
N (τ ) dXN (τ )
=
Z t
0
Y (τ ) dX(τ ) + 1
2
Xn
i=1
Z t
0
∂if

Z(τ )

hZi
, Xi(dτ ),
where the convergence is in P-probability uniformly for t in compacts.
(iii) Show that (8.1.6) continues to hold for f ∈ C
2
(R
n; R) when the integral on the right hand side is interpreted using the extension of Stratonovich
integration develope
Exercise 8.1.11. Let X, Y ∈ S(P; R). In connection with Remark 8.1.7,
it is interesting to examine whether it is sufficient to mollify only X when
defining the Stratonovich integral of Y with respect to X.
(i) Show R 1
0
Y ([τ ]N ) dXN (τ ) tends in P-probability to R 1
0
Y (τ ) dX(τ ).
(ii) Define ψN (t) = 1 − 2
N (t − [t]N ), set ZN (t) = R t
0
ψN (τ ) dY (τ ), and
show that
Z 1
0
Y (τ ) dXN (τ ) −
Z 1
0
Y ([τ ]N ) dXN (τ ) =
2
XN −1
m=0

∆N
mX
∆N
mZN

.
(iii) Show that
2
XN −1
m=0

∆N
mX
∆N
mZN

−
Z 1
0
ψN (t)hX, Y i(dt) −→ 0 in P-probability.
(iv) Show that for any Lebesgue integrable function α : [0, 1] −→ R,
R 1
0
ψN (τ )α(τ ) dτ tends to 1
2
R 1
0
α(τ ) dτ .
(v) Under the condition that hX, Y i(dt) = β(t) dt, where β : [0,∞) ×
Ω −→ [0,∞) is a progressively measurable function, use the preceding to see
that R 1
0
Y (τ ) dXN (τ ) tends in P-probability to R 1
0
Y (τ ) ◦ dX(τ ).
Exercise 8.1.12. One place where Stratonovich’s theory really comes into
its own is when it comes to computing the determinant of the solution
to a linear stochastic differential equation. Namely, suppose that A =
((Aij ))1≤i,j≤n ∈ S
P; Hom(R
n; R
n)

(i.e., Aij ∈ S(P; R) for each 1 ≤ i, j ≤
n), and assume that X ∈ S
P; Hom(R
n; R
n)

satisfies dX(t) = X(t) ◦ dA(t)
in sense that
dXij (t) = Xn
k=1
Xik(t) ◦ dAkj (t) for all 1 ≤ i, j ≤ n.
Show that
det
X(t)

= det
X(0)
e
Trace
A(t)−


\end{document}